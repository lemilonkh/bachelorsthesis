\documentclass[
        a4paper,     % Format A4
        titlepage,   % mit Titelseite
        twoside,     % zweiseitig
        parskip      % mit Durchschuss
                     % (= Abstand zwischen Absätzen, statt Einrückung)
        ]{scrartcl}  % KOMA-Script Grundklasse     texdoc scrguide

%\usepackage{ngerman}             % Deutsche Sprache, neue RS   texdoc germdoc (?)
\usepackage[T1]{fontenc}          % Schriftkodierung mit Umlauten
\usepackage{textcomp,amsmath}     % Mathezeichen etc.
\usepackage{graphicx}             % Graphiken einbinden
\usepackage{hyperref}             % Hyperlinks fuer TOC, Glossar, etc.

% bibtex
\usepackage{url}
\bibliographystyle{plaindin}      % BibTeX Styles nach Norm DIN 1505

\usepackage[xindy,acronym,toc,acronym]{glossaries} % Glossar definieren

\setglossarystyle{list} % or altlist, listgroup, listhypergroup

% glossary entries
\newglossaryentry{business}{
  name=business,
  description={An economic entity that is to be represented as a node in the business graph}
}
\newglossaryentry{lambda}{
  name={\ensuremath{\lambda}},
  description={Lambda!},
  sort=lambda
}
\newacronym{bgn}{BGN}{business graph networks}

\makeglossaries


\titlehead{
  \includegraphics{hpi_logo_cmyk_wb_sl2}
}
\subject{Bachelorarbeit}
\title{
  User-aided Pattern Search and Analysis on Business Graphs
  \\ \bigskip
  \large{Nutzergestuetzte Graphanalyse und Mustersuche auf Unternehmensgraphen}
}
\author{Milan Gruner\\{\small{\url{milangruner@gmail.com}}}}
\date{Eingereicht am <TBD>}
\publishers{
  Fachgebiet Informationssysteme \\
  Betreuung: Prof. Dr. Felix Naumann, Michael Loster, Toni Gruetze
}

\pagestyle{headings}    % Seitenstil mit Kapitelüberschriften in der Kopfzeile

\begin{document}

  \maketitle    %  Titelseite erzeugen
  \clearpage % neue Seite (war \cleardoublepage)

  \section*{Abstract}
  Costructing a graph made up of thousands of businesses may be hard, but actually making sense of it is a lot harder.
  With huge amounts of data potentially being integrated into the data lake every day, automatic methods for finding interesting spots in the graph are needed.
  This paper discusses different approaches that can be taken to extract useful knowledge from such a graph.

  \clearpage
  \tableofcontents
  %\listoffigures
  %\listoftables

	%\glsaddall
  %\printglossary
  %\printglossary[type=\acronymtype
  %\Gls{business} \\
  %\gls{lambda} \\
  %\gls{bgn} \\
  %\cite{bloom70space-time}

  \section{Introduction}
    \subsection{Glossary}
    \subsection{Motivation}
    \subsection{Understanding risk analysis on graphs}
    \subsection{Used techniques and related works}
    \pagebreak

  \section{Data structures for business entities}
    \subsection{Graph encoding for column family storage}
    \subsection{The $subject$ data structure}
    \subsection{A versioning scheme that stands the test of time}
    \pagebreak

  \section{Architecture}
    \subsection{Job and Data Management}
      \subsubsection{Modularizing Spark jobs}
      \subsubsection{Coordinating Spark jobs from NodeJS}
      \subsubsection{Managing Cassandra tables from NodeJS}
      \subsubsection{Data flow using column family storage}
    \subsection{Using Apache Spark and Cassandra for Graph Analysis}
      \subsubsection{Writing efficient Spark GraphX code}
      \subsubsection{Optimizing Cassandra data structures for Graph Processing}
      \pagebreak

  \section{Pattern Search}
    \subsection{Discerning patterns from randomness}
    \subsection{Operating on graph diffs}
    \subsection{Pattern types and their applications}
    \pagebreak

  \section{Pattern Analysis}
    \subsection{User-aided approaches for Pattern Categorization}
    \subsection{Pattern importance measures}
    \subsection{Machine Learning Models for analyzing user feedback}
    \pagebreak

  \section{Graph Summarization}
    \subsection{What users actually want to see}
    \subsection{Compressing graph information to the bare minimum}
    \subsection{Presenting graph data appealingly}
    \pagebreak

  \section{Lessons learned}
    \subsection{Benchmarks and Experiments}
    \subsection{Design decisions and trade-offs}
    \subsection{Technical challenges}
    \pagebreak

  \section{Literature}

  \clearpage
  \bibliography{references}

\end{document}
