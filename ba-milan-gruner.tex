\documentclass[
        a4paper,     % Format A4
        titlepage,   % mit Titelseite
        twoside,     % zweiseitig
        parskip      % mit Durchschuss
                     % (= Abstand zwischen Absätzen, statt Einrückung)
        ]{scrartcl}  % KOMA-Script Grundklasse     texdoc scrguide

%\usepackage{ngerman}             % Deutsche Sprache, neue RS   texdoc germdoc (?)
\usepackage[T1]{fontenc}          % Schriftkodierung mit Umlauten
\usepackage{textcomp,amsmath}     % Mathezeichen etc.
\usepackage{graphicx}             % Graphiken einbinden
\usepackage{hyperref}             % Hyperlinks fuer TOC, Glossar, etc.

% bibtex
\usepackage{url}
\bibliographystyle{plaindin}      % BibTeX Styles nach Norm DIN 1505

\usepackage[xindy,acronym,toc,acronym]{glossaries} % Glossar definieren

\setglossarystyle{list} % or altlist, listgroup, listhypergroup

% glossary entries
\newglossaryentry{business}{
  name=business,
  description={An economic entity that is to be represented as a node in the business graph}
}
\newglossaryentry{lambda}{
  name={\ensuremath{\lambda}},
  description={Lambda!},
  sort=lambda
}
\newacronym{bgn}{BGN}{business graph networks}

\makeglossaries


\titlehead{
  \includegraphics{hpi_logo_cmyk_wb_sl2}
}
\subject{Bachelorarbeit}
\title{
  User-aided Pattern Search and Analysis on Business Graphs
  \\ \bigskip
  \large{Nutzergestuetzte Graphanalyse und Mustersuche auf Unternehmensgraphen}
}
\author{Milan Gruner\\{\small{\url{milangruner@gmail.com}}}}
\date{Eingereicht am <TBD>}
\publishers{
  Fachgebiet Informationssysteme \\
  Betreuung: Prof. Dr. Felix Naumann, Michael Loster, Toni Gruetze
}

\pagestyle{headings}    % Seitenstil mit Kapitelüberschriften in der Kopfzeile

\begin{document}

  \maketitle    %  Titelseite erzeugen
  \clearpage % neue Seite (war \cleardoublepage)

  \section*{Abstract}
  { \huge
    Costructing a graph made up of thousands of businesses may be hard, but actually making sense of it is a lot harder.
    With huge amounts of data potentially being integrated into the data lake every day, automatic methods for finding interesting spots in the graph are needed.
    This paper discusses different approaches that can be taken to extract useful knowledge from such a graph.
  }

  \clearpage
  \tableofcontents
  %\listoffigures
  %\listoftables

	%\glsaddall
  %\printglossary
  %\printglossary[type=\acronymtype
  %\Gls{business} \\
  %\gls{lambda} \\
  %\gls{bgn} \\
  %\cite{bloom70space-time}

  \section{Introduction}
    This paper describes the basic approach of Pattern Search and Analysis
    on the graphs generated as a part of the Ingestion bachelor project at HPI Potsdam.
    It deals with the experiments I performed to evaluate certain algorithms and approaches to summarize
    automatically generated graphs in the ball park of tens of millions of nodes in
    a large and mostly unconnected and sparse graph.
    It contains businesses, places and persons and can be augmented freely by adding
    additional data sources (or manual data input) of your own. The data is visualized and controllable from the
    Curation interface, which was a project developed alongside the Ingestion pipeline.
    Curation generates graphs on-the-fly that contain the statistics data that was written
    as a side effect of the numerous Spark jobs in the Ingestion pipeline and gives
    a lot of insight into what's going on in the "data lake".
    All of the experiments contained in this paper were conducted using either
    the Curation interface, the Apache Zeppelin interactive web shell or the Spark shell.

    Most of the topics in this paper deal with graph-related topics, so the Spark GraphX framework
    (in Scala) will mostly be used to describe the algorithms that were employed,
    but it can be freely seen as functional pseudo code and adapted to other popular
    graph processing frameworks (such as Giraph etc.) % TODO more examples
    For UI-related or user-oriented algorithms, JavaScript (or JSX, so partly using React/ Redux etc.)
    is the listing language, as visual problems are more easily expressed in this
    tree-oriented but still functional style (and also because it's used in Curation). % TODO remove?
    Whereever possible, ES2016 Syntax is used for the brevity and the similarity
    to the functional approach of Scala and the rest of the example code.

    Also one of the purposes of the Ingestion and Curation projects was to enable
    the execution of modern text mining, deduplication and parallel data analysis
    to the german business landscape. This means that some of the examples will be
    in German (but they will be translated if needed for understanding the technique).

    \subsection{Usage Scenarios of Ingestion, Curation and [this paper]} % TODO add project name
    The process of classic risk analysis which is still majorly employed in today's banks is mostly
    a model-oriented process that heavily relies on speculation or predictions.\\
    The techniques that Ingestion, Curation and this paper (TODO: project name?)
    offer are universally usable to analyze huge amounts of structured or unstructured data
    in the form of WikiData, JSON, CSV or similar data as well as newspaper articles,
    messages (eMail, instant or otherwise).\\
    The results of this can either be viewed as a graph or a relational model
    using the conversion algorithms briefly discussed later in this paper. % TODO link subsection
    Because [this system] has automatic as well as user-driven processes  % TODO add project name
    that control the flow, categorization and importance grading % TODO better word
    of all the data in the data lake, it may be effectively deployed in a bank
    or similar institution as a data collection, curation and analysis software stack.\\
    It is meant to be heavily extensible and allows the simple integration of new data sources % TODO really possible in the end?
    as well as algorithms and data structures in general. The pipeline itself is generally very modular
    and is composed of many stand-alone Spark jobs that mostly read and write from/ to the Cassandra database.
    For more information on this technique, see [Nils' paper on Cassandra]. % TODO link

    \subsection{Used techniques and assumptions}
    TODO

  \section{Data structures for business entities}
    \subsection{Graph encoding for column family storage}
    \subsection{The $subject$ data structure}
    \subsection{A versioning scheme that stands the test of time}

  \section{Architecture (opt.)}
    \subsection{Job and Data Management}
      \subsubsection{Modularizing Spark jobs}
      \subsubsection{Coordinating Spark jobs from NodeJS}
      \subsubsection{Managing Cassandra tables from NodeJS}
      \subsubsection{Data flow using column family storage}
    \subsection{Using Apache Spark and Cassandra for Graph Analysis}
      \subsubsection{Motivation: Why Spark, Cassandra and GraphX?}
      \subsubsection{Writing efficient Spark GraphX code}
      \subsubsection{Optimizing Cassandra data structures for Graph Processing}
    \pagebreak

  \section{Graph Summarization}
    \subsection{What users actually want to see}
    \subsection{Compressing graph information to the bare minimum}
    \subsection{Presenting graph data appealingly}
    \subsection{Related work}
    \pagebreak

  \section{Pattern Search}
    \subsection{Discerning patterns from randomness}
    \subsection{Operating on graph diffs}
    \subsection{Pattern types and their applications}
    \subsection{Related work}
    \pagebreak

  \section{Pattern Analysis}
    \subsection{User-aided approaches for Pattern Categorization}
    \subsection{Pattern importance measures}
    \subsection{Machine Learning Models for analyzing user feedback}
    \subsection{Related work}
    \pagebreak

  \section{Lessons learned}
    \subsection{Benchmarks and Experiments}
    \subsection{Design decisions and trade-offs}
    \subsection{Technical challenges}
    \pagebreak

  \section{Literature}

  \clearpage
  \bibliography{references}

\end{document}
